\documentclass[10pt,a4paper]{article}

\usepackage{datetime}
\usepackage{numprint}
\usepackage{palatino}
\usepackage{authblk}
\usepackage[margin=0.75in]{geometry}
\usepackage{hyperref}
\usepackage{graphicx}
\usepackage{titlesec}
\usepackage{listings}
\usepackage[english]{babel}
\usepackage[
	backend=biber,
	style=numeric,
]{biblatex}
\addbibresource{refs.bib}

\hypersetup{%
    pdfborder = {0 0 0}
}

\setlength{\parindent}{2em}
%\setlength{\parskip}{1em}
\renewcommand{\baselinestretch}{1.0}

\begin{document}

\nplpadding{2}

\title{Malware Analysis Report: ``Practical2.exe''\\ \vspace{-8pt} {\large CAP6137 Malware Reverse Engineering: P0x02}}
\author{{Naman Arora \\ \vspace{-10pt}\small \href{mailto:naman.arora@ufl.edu}{naman.arora@ufl.edu}}}
\date{\today}

\maketitle
\newpage
\tableofcontents
\newpage
\section{Executive Summary}

\section{Static Analysis: Primary Executable}
\subsection{Basic Identification}
\begin{center}
	\begin{tabular}{c | c}
		Attribute & Value\\
		\hline
		\hline
		Bits & 32\\
		Endianess & Little\\
		Operating System & Microsoft Windows\\
		\hline
		Class & PE32\\
		Subsystem & Windows CUI\\
		\hline
		Size & 1446912\\
		Compiler Timestamp & Thu Dec 10 02:47:43 2020\\
		Compiler & Visual Studio\\
		SHA256 Hash & 9633d0564a2b8f1b4c6e718ae7ab48be921d435236a403cf5e7ddfbfd4283382\\
		\hline
	\end{tabular}
\end{center}

\subsection{Malware Sample Family Identification}
\begin{figure}[!htbp]% [!hb] forces image to be placed at that position
	\centering
	\includegraphics[width=\columnwidth]{pics/family.png}
	\caption{VirusTotal: VirusTotal Scan}
	\label{family}
\end{figure}
The given PE file, on being uploaded to VirusTotal, is identified as a variant of \textit{AveMariaRAT} family (Fig. \ref{family})
As seen later in the \textit{dynamic analysis} section, another in-memory PE when dumped and analysed on VirusTotal, is identified to belong to \textit{WarZoneRAT} family.

\subsection{PE Sections}
\begin{figure}[!htbp]% [!hb] forces image to be placed at that position
	\centering
	\includegraphics[width=\columnwidth]{pics/section_entropy.png}
	\caption{Rizin: Section-wise Entropy}
	\label{entropy}
\end{figure}

	\subsubsection{The \textit{.text}, \textit{.rdata}, \textit{.idata}, \textit{.rsrc} and \textit{.reloc} sections}
	These commonly found PE sections within the executable show no peculiar characteristics in terms of entropy, virtual sizes and permissions.
	\subsubsection{The \textit{.data} Section}
	This section, although not peculiar either, on static analysis reveals that it is referenced in the identified main function.
	On further analysis of the function, a unpacking loop is encountered thus hinting towards the section being the store of packed data.
	\subsubsection{The \textit{.tls} Section}
	Presence of this section generally hints towards thread execution before \textit{entrypoint} is reached in the context of malicious binaries.
	This binary, however, shows no such execution. Thus, the reason for the presence of this section cannot be corroborated during the current analysis.
	\subsubsection{The \textit{.00cfg} Section}
	The presence of this unusual section (\textit{Control Flow Guard}) seems to be explained as an artifact of the \textit{Visual studio compiler}.
	This guess is supported by
	\begin{itemize}
		\item Very small size of the section \textit{0x200}.
		\item Almost all bytes being zeros.
		\item All the references to this section (Fig. \ref{cfg}) seem to originate from \textit{Ghidra} identified library functions with exception to one which does not show much promise on followup.
	\end{itemize}
	\begin{figure}[!htbp]% [!hb] forces image to be placed at that position
		\centering
		\includegraphics[width=\columnwidth]{pics/cfg.png}
		\caption{Ghidra: references to the \textit{.00cfg} section}
		\label{cfg}
	\end{figure}

\subsection{A case for Packing}
\begin{figure}[!htbp]% [!hb] forces image to be placed at that position
	\centering
	\includegraphics[width=\columnwidth]{pics/unpacking.png}
	\caption{Ghidra: Disassembly of unpacking}
	\label{unpacking}
\end{figure}
A very strong case for packing can be made for this binary given the following observations,
\begin{itemize}
	\item The identified \textit{main} function exhibits a series of byte operations on data pointed to by the \textit{.data} section.
	\item Immediately preceding the manipulations, a call to \textit{VirtualAlloc} can be intercepted.
	\item The manipulated bytes from \textit{.data} section are stored in the allocated memory section.
	\item After the said manipulations, the memory section is called as a function.
	\item The said allocated section, on analysis and after being manipulated, exhibits a presence of \textit{shell code} and a \textit{PE} header preceding code at repeatedly reproducible offsets and sizes.
\end{itemize}

\subsection{Interesting Imports}
\begin{figure}[!htbp]% [!hb] forces image to be placed at that position
	\centering
	\includegraphics[width=\columnwidth]{pics/imports_parent.png}
	\caption{Ghidra: Imports tree}
	\label{imports_parent}
\end{figure}
	\subsubsection{Imports from \textit{Kernel32.dll}}
	Imports like \textit{VirtualAlloc and VirtualFree} in combination with \textit{VirtualProtect} strongly indicate runtime memory injection preceding change in injected region's permissions to \textit{executable}.
	Presence of \textit{FreeConsole} seems to corroborate the assumption that this is a \textit{CUI} program, given this function is used to unlink from the parent process.

	\subsection{Imports from \textit{user32.dll}}
	An import from this library, \textit{viz.}, \textit{MessageBoxA} is peculiar. This is due to the fact that, in \textit{main} function, the permissions of memory containing code for this import is updated from \textit{PAGE\_EXECUTE\_READ} to
	\textit{PAGE\_EXECUTE\_READWRITE} and is subsequently the code is replaced with a \textit{return 0x10000} call.
	This function is then invoked multiple times during the unpacking process and the string \textit{``pbstrPath != 0 \&\& ppTypeLib != 0''} is pushed as twice arguments. The reason behind this could not be identified during this analysis (Fig \ref{unpacking}).

\section{Static Analysis: Dynamically Unpacked Shell Code}
\subsection{Basic Identification}
\begin{center}
	\begin{tabular}{c | c}
		Attribute & Value\\
		\hline
		\hline
		Bits & 32\\
		Endianess & Little\\
		Operating System & Microsoft Windows\\
		\hline
		Class & PE32\\
		Subsystem & Windows CUI\\
		\hline
		Size & 1446912\\
		Compiler Timestamp & Thu Dec 10 02:47:43 2020\\
		Compiler & Visual Studio\\
		SHA256 Hash & 9633d0564a2b8f1b4c6e718ae7ab48be921d435236a403cf5e7ddfbfd4283382\\
		\hline
	\end{tabular}
\end{center}

\subsection{Sample Family Identification}

\subsection{Shell Code Sections}
	\subsubsection{The \textit{.text}, \textit{.rdata}, \textit{.idata}, \textit{.rsrc} and \textit{.reloc} sections}
	These commonly found PE sections within the executable show no peculiar characteristics in terms of entropy, virtual sizes and permissions.
	\subsubsection{The \textit{.data} Section}
	This section, although not peculiar either, on static analysis reveals that it is referenced in the identified main function.
	On further analysis of the function, a unpacking loop is encountered thus hinting towards the section being the store of packed data.
	\subsubsection{The \textit{.tls} Section}
	Presence of this section generally hints towards thread execution before \textit{entrypoint} is reached in the context of malicious binaries.
	This binary, however, shows no such execution. Thus, the reason for the presence of this section cannot be corroborated during the current analysis.
	\subsubsection{The \textit{.00cfg} Section}
	The presence of this unusual section (\textit{Control Flow Guard}) seems to be explained as an artifact of the \textit{Visual studio compiler}.
	This guess is supported by
	\begin{itemize}
		\item Very small size of the section \textit{0x200}.
		\item Almost all bytes being zeros.
		\item All the references to this section (Fig. \ref{cfg}) seem to originate from \textit{Ghidra} identified library functions with exception to one which does not show much promise on followup.
	\end{itemize}
	\begin{figure}[!htbp]% [!hb] forces image to be placed at that position
		\centering
		\includegraphics[width=\columnwidth]{pics/cfg.png}
		\caption{Ghidra: references to the \textit{.00cfg} section}
		\label{cfg}
	\end{figure}

\subsection{Interesting Imports}

\section{Static Analysis: Dynamically Unpacked PE Executable}
\subsection{Basic Identification}
\begin{center}
	\begin{tabular}{c | c}
		Attribute & Value\\
		\hline
		\hline
		Bits & 32\\
		Endianess & Little\\
		Operating System & Microsoft Windows\\
		\hline
		Class & PE32\\
		Subsystem & Windows CUI\\
		\hline
		Size & 1446912\\
		Compiler Timestamp & Thu Dec 10 02:47:43 2020\\
		Compiler & Visual Studio\\
		SHA256 Hash & 9633d0564a2b8f1b4c6e718ae7ab48be921d435236a403cf5e7ddfbfd4283382\\
		\hline
	\end{tabular}
\end{center}

\subsection{Sample Family Identification}

\subsection{PE Sections}
	\subsubsection{The \textit{.text}, \textit{.rdata}, \textit{.idata}, \textit{.rsrc} and \textit{.reloc} sections}
	\subsubsection{The \textit{.data} Section}
	\subsubsection{The \textit{.tls} Section}
	\subsubsection{The \textit{.00cfg} Section}

\subsection{Interesting Imports}

\section{Static Analysis: Dynamically Unpacked DLL}
\subsection{Basic Identification}
\begin{center}
	\begin{tabular}{c | c}
		Attribute & Value\\
		\hline
		\hline
		Bits & 32\\
		Endianess & Little\\
		Operating System & Microsoft Windows\\
		\hline
		Class & PE32\\
		Subsystem & Windows CUI\\
		\hline
		Size & 1446912\\
		Compiler Timestamp & Thu Dec 10 02:47:43 2020\\
		Compiler & Visual Studio\\
		SHA256 Hash & 9633d0564a2b8f1b4c6e718ae7ab48be921d435236a403cf5e7ddfbfd4283382\\
		\hline
	\end{tabular}
\end{center}

\subsection{Sample Family Identification}

\subsection{Sections}
	\subsubsection{The \textit{.text}, \textit{.rdata}, \textit{.idata}, \textit{.rsrc} and \textit{.reloc} sections}
	\subsubsection{The \textit{.data} Section}
	\subsubsection{The \textit{.tls} Section}
	\subsubsection{The \textit{.00cfg} Section}

\subsection{Interesting Imports}

\section{Dynamic Analysis: Primary Executable}
		\subsection{Network Based Analysis}
			\subsubsection{External domains contacted}
			\subsubsection{Internet Protocols Used}
			\subsubsection{Contents of Communication}

		\subsection{File System Based Analysis}
				\subsubsection{File System Changes}
				\subsubsection{Windows Registry Changes}
		\subsection{Memory Forensics}
				\subsubsection{A case for Code Injection}
				\subsubsection{Memory region analysis}

\section{Indicators of Compromise}
			\subsection{Network Based}
			\subsection{Host Based}

\newpage
\section{Appendix A: Memory Dump string analysis screenshots}

\newpage
\printbibliography
\end{document}