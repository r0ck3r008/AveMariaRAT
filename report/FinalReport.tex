\documentclass[10pt,a4paper]{article}

\usepackage{datetime}
\usepackage{numprint}
\usepackage{palatino}
\usepackage{authblk}
\usepackage[margin=0.75in]{geometry}
\usepackage{hyperref}
\usepackage{graphicx}
\usepackage{titlesec}
\usepackage{listings}
\usepackage[english]{babel}
\usepackage[
	backend=biber,
	style=numeric,
]{biblatex}
\addbibresource{refs.bib}

\hypersetup{%
    pdfborder = {0 0 0}
}

\setlength{\parindent}{2em}
%\setlength{\parskip}{1em}
\renewcommand{\baselinestretch}{1.0}

\begin{document}

\nplpadding{2}

\title{Malware Analysis Report: ``Practical2.exe''\\ \vspace{-8pt} {\large CAP6137 Malware Reverse Engineering: P0x02}}
\author{{Naman Arora \\ \vspace{-10pt}\small \href{mailto:naman.arora@ufl.edu}{naman.arora@ufl.edu}}}
\date{\today}

\maketitle
\newpage
\tableofcontents
\newpage
\section{Executive Summary}
The provided binary is a PE executable 32-bit Microsoft Windows platform. This is certainly a malicious executable and has string resemblance to \textit{AveMariaRAT} and \textit{WarZoneRAT} trojan families. The malware is obfuscated to thwart analysis in both static and dynamic phases.

The primary malware on execution unpacks an intermediary \textit{shell-code} stage as well as another PE and writes them to its own memory. The execution then passes on to the said \textit{shell-code} which then unpacks another PE, this time a \textit{Dynamic Linked Library, DLL}, again within its own memory.

The malware \cite{warzone}, largely, is capable of,
\begin{itemize}
	\item Soliciting remote desktop connections \textit{RDP}
	\item Bypassing important Windows security features like \textit{UAC}, and \textit{Defender}
	\item Remotely monitoring using Webcam, KeyLogger, Process manager etc.
	\item Activating reverse proxy, poking hole in internal network for external access
	\item Upload, download and execution of files from internet on victim
\end{itemize}

\textit{IOCs} mentioned towards the end of this report can be leveraged to detect this malware in transit on network or on file system.

\newpage

\section{Static Analysis: Primary Executable}
\subsection{Basic Identification}
\begin{center}
	\begin{tabular}{c | c}
		Attribute & Value\\
		\hline
		\hline
		Bits & 32\\
		Endianess & Little\\
		Operating System & Microsoft Windows\\
		\hline
		Class & PE32\\
		Subsystem & Windows CUI\\
		\hline
		Size & 1446912\\
		Compiler Timestamp & Thu Dec 10 02:47:43 2020\\
		Compiler & Visual Studio\\
		SHA256 Hash & 9633d0564a2b8f1b4c6e718ae7ab48be921d435236a403cf5e7ddfbfd4283382\\
		\hline
	\end{tabular}
\end{center}

\subsection{Malware Sample Family Identification}
\begin{figure}[!htbp]% [!hb] forces image to be placed at that position
	\centering
	\includegraphics[width=\columnwidth]{pics/family.png}
	\caption{VirusTotal: VirusTotal Scan}
	\label{family}
\end{figure}
The given PE file, on being uploaded to VirusTotal, is identified as a variant of \textit{AveMariaRAT} \cite{avemaria} family (Fig. \ref{family})
As seen later in the \textit{dynamic analysis} section, another in-memory PE when dumped and analysed on VirusTotal, is identified to belong to \textit{WarZoneRAT} \cite{warzone} family.

\subsection{PE Sections}
\begin{figure}[!htbp]% [!hb] forces image to be placed at that position
	\centering
	\includegraphics[width=\columnwidth]{pics/section_entropy.png}
	\caption{Rizin: Section-wise Entropy}
	\label{entropy}
\end{figure}

	\subsubsection{The \textit{.text}, \textit{.rdata}, \textit{.idata}, \textit{.rsrc} and \textit{.reloc} sections}
	These commonly found PE sections within the executable show no peculiar characteristics in terms of entropy, virtual sizes and permissions.
	\subsubsection{The \textit{.data} Section}
	This section, although not peculiar either, on static analysis reveals that it is referenced in the identified main function.
	On further analysis of the function, a unpacking loop is encountered thus hinting towards the section being the store of packed data.
	\subsubsection{The \textit{.tls} Section}
	Presence of this section generally hints towards thread execution before \textit{entrypoint} is reached in the context of malicious binaries.
	This binary, however, shows no such execution. Thus, the reason for the presence of this section cannot be corroborated during the current analysis.
	\subsubsection{The \textit{.00cfg} Section}
	The presence of this unusual section (\textit{Control Flow Guard}) seems to be explained as an artifact of the \textit{Visual studio compiler}.
	This guess is supported by
	\begin{itemize}
		\item Very small size of the section \textit{0x200}.
		\item Almost all bytes being zeros.
		\item All the references to this section (Fig. \ref{cfg}) seem to originate from \textit{Ghidra} identified library functions with exception to one which does not show much promise on followup.
	\end{itemize}
	\begin{figure}[!htbp]% [!hb] forces image to be placed at that position
		\centering
		\includegraphics[width=\columnwidth]{pics/cfg.png}
		\caption{Ghidra: references to the \textit{.00cfg} section}
		\label{cfg}
	\end{figure}

\subsection{A case for Packing}
\begin{figure}[!htbp]% [!hb] forces image to be placed at that position
	\centering
	\includegraphics[width=\columnwidth]{pics/unpacking.png}
	\caption{Ghidra: Disassembly of unpacking}
	\label{unpacking}
\end{figure}
A very strong case for packing can be made for this binary given the following observations,
\begin{itemize}
	\item The identified \textit{main} function exhibits a series of byte operations on data pointed to by the \textit{.data} section.
	\item Immediately preceding the manipulations, a call to \textit{VirtualAlloc} can be intercepted.
	\item The manipulated bytes from \textit{.data} section are stored in the allocated memory section.
	\item After the said manipulations, the memory section is called as a function.
	\item The said allocated section, on analysis and after being manipulated, exhibits a presence of \textit{shell code} and a \textit{PE} header preceding code at repeatedly reproducible offsets and sizes.
\end{itemize}

\subsection{Interesting Imports}
\begin{figure}[!htbp]% [!hb] forces image to be placed at that position
	\centering
	\includegraphics[width=\columnwidth]{pics/imports_parent.png}
	\caption{Ghidra: Imports tree}
	\label{imports_parent}
\end{figure}
	\subsubsection{Imports from \textit{Kernel32.dll}}
	Imports like \textit{VirtualAlloc and VirtualFree} in combination with \textit{VirtualProtect} strongly indicate runtime memory injection preceding change in injected region's permissions to \textit{executable}.
	Presence of \textit{FreeConsole} seems to corroborate the assumption that this is a \textit{CUI} program, given this function is used to unlink from the parent process.

	\subsubsection{Imports from \textit{user32.dll}}
	An import from this library, \textit{viz.}, \textit{MessageBoxA} is peculiar. This is due to the fact that, in \textit{main} function, the permissions of memory containing code for this import is updated from \textit{PAGE\_EXECUTE\_READ} to
	\textit{PAGE\_EXECUTE\_READWRITE} and is subsequently the code is replaced with a \textit{return 0x10000} call.
	This function is then invoked multiple times during the unpacking process and the string \textit{``pbstrPath != 0 \&\& ppTypeLib != 0''} is pushed as twice arguments. The reason behind this could not be identified during this analysis (Fig \ref{unpacking}).

\section{Dynamic leading to Static Analysis: Unpacked Shell Code}
\subsection{Basic Identification}
\begin{center}
	\begin{tabular}{c | c}
		Attribute & Value\\
		\hline
		\hline
		Bits & 32\\
		Endianess & Little\\
		\hline
		Class & Raw Binary\\
		\hline
		Size & 1343\\
		Compiler & Visual Studio (Likely)\\
		SHA256 Hash & 7203a68d0fcbde21f4005f45b14ff9ee625e16dfcf936fd82743d6bf88f76b91\\
		\hline
	\end{tabular}
\end{center}

\subsection{Sample Family Identification}
\begin{figure}[!htbp]% [!hb] forces image to be placed at that position
	\centering
	\includegraphics[width=\columnwidth]{pics/shellcode-family.png}
	\caption{VirusTotal: Shell Code Family}
	\label{shellcode}
\end{figure}
The \textit{shell code}, on submission to \textit{VirusTotal}, was identified as \textit{generic malicious shell code}.
Although, a comment on the submission to the \textit{VirusTotal} from community seems to suggest this \textit{shell code} is generated by \textit{BC-SECURITY/Empire} \cite{empire} post exploitation framework. (Fig. \ref{shellcode}).

\subsection{Shell Code Sections}
There are \textbf{\textit{four}} distinctly identifiable sections of the shell code, \textit{viz.}
\begin{itemize}
	\item \textit{Call/Pop} to get \textit{EIP} (Fig \ref{callpop}).
	\item Call to \textit{build\_IAT\_and\_Jump} function at offset \textit{0x2d} by pushing the pointers to the end of shell code and the \textit{WarZone RAT} binary \textit{Discussed later}, \textit{EIP} and some other as of yet unidentified arguments (Fig. \ref{shellcode-push}).
	\item The \textit{build\_IAT\_and\_Jump} function at offset \textit{0x2d} which possibly allocates memory for another injection of a \textit{DLL (Discussed later)}, resolves the imports for the \textit{WarZone RAT binary}, executes a function from \textit{DLL} and eventually causes an exception to execute the \textit{WarZone RAT} OEP.
	\item The \textit{getFuncNames} function at offset \textit{0x467} which possibly un-hashes the imports for the shell code itself and is called \textit{six} times in total.
\end{itemize}

\subsection{Interesting Imports}
Although the shell code does not import anything due to it being \textit{Position Independent Code}, it certainly un-hashes \textit{\textbf{six}} imports in particular by using possibly the previously mentioned \textit{getFuncNames} function.
These imports are,
\begin{itemize}
	\item \textit{LoadLibraryA}
	\item \textit{GetProcAddress}
	\item \textit{VirtualAlloc}
	\item \textit{VirtualProtect}
	\item \textit{ZwFlushInstructionCache}
	\item \textit{GetNativeSystemInfo}
\end{itemize}

\section{Dynamic leading to Static Analysis: Unpacked PE Executable}
\subsection{Basic Identification}
\begin{center}
	\begin{tabular}{c | c}
		Attribute & Value\\
		\hline
		\hline
		Bits & 32\\
		Endianess & Little\\
		Operating System & Microsoft Windows\\
		\hline
		Class & PE32\\
		Subsystem & Windows CUI\\
		\hline
		Size & 1383640\\
		Compiler Timestamp & 2020-08-29 07:01:59\\
		Compiler & Visual Studio\\
		SHA256 Hash & 37a5c9162c834ecf877a9461e29b5adba92cbcbbe07fe56685e4f7982d1a9bc8\\
		\hline
	\end{tabular}
\end{center}

\subsection{Sample Family Identification}
This is a malicious PE extracted from the memory of primary PE after being unpacked.
The submission on \textit{VirusTotal} shows this belongs to the \textit{WarZonRAT} \cite{warzone} family of trojans.
Moreover, presence of the string \textit{warzone160} and others link it to the family to a high degree of confidence (Fig. \ref{warzone}).

\subsection{Packing and Further Obfuscation}
This dumped binary does not show signs of being obfuscated further in terms of being packed.
This assumption is supported by observations,
\begin{itemize}
	\item Presence of a multitude of imports that generally would be obfuscated.
	\item Presence of a multitude of ASCII strings and functions.
	\item Nominal range of entropy of individual sections.
\end{itemize}

\subsection{Interesting Imports}
	\subsubsection{Imports from \textit{bcrypt.dll}}
	This library shows the imports \textit{BCryptDecrypt, BCryptGenerateSymmetricKey, BCryptOpenAlgorithmProvider and BCryptSetProperty} which hint towards symmetric key generation and decryption of data. Notably, absence of an \textit{encrypt} counterpart along with this being a \textit{RAT} leads to suspicion that something encrypted is received over network activity which then is decrypted.

	\subsubsection{Import from \textit{urlmon.dll}}
	An import of \textit{URLDownloadToFile} indicates a downloader like behavior.

	\subsubsection{Imports from \textit{shell32.dll}}
	Imports like \textit{SHCreateDirectory, ShellExecuteA, SHGetFolderPath, etc} indicate towards filesystem manipulation behavior as well as executing some other OS command.

	\subsubsection{Imports from \textit{netapi32.dll}}
	Imports \textit{NetLocalGroupAddMembers and NetUserAdd} hint towards a backdoor like behavior.

	\subsubsection{Imports from \textit{advapi32.dll}}
	Imports like \textit{RegCreateKey, RegSetValue, RegCloseKey, AdjustTokenPrivileges and GetTokenInformation} hint towards registry action \textit{(also later corroborated by dynamic analysis)} as well as privilege escalation.

\section{Dynamic leading to Static Analysis: Unpacked DLL}
\subsection{Basic Identification}
\begin{center}
	\begin{tabular}{c | c}
		Attribute & Value\\
		\hline
		\hline
		Bits & 32\\
		Endianess & Little\\
		Operating System & Microsoft Windows\\
		\hline
		Class & PE32\\
		Subsystem & Windows CUI\\
		\hline
		Size & 1383640\\
		Compiler Timestamp & 2020-08-29 07:01:59\\
		Compiler & Visual Studio\\
		SHA256 Hash & a0e0bdb288eb7bf5585cbe101c30b892e0d5d916fa9f2a90d2059d6c8382be3e\\
		\hline
	\end{tabular}
\end{center}

\subsection{Sample Family Identification}
The extracted binary is a \textit{DLL} linked to both \textit{AveMariaRAT} \cite{avemaria} as well as \textit{WarZoneRAT} \cite{warzone}, as illustrated by submission to \textit{VirusTotal} (Fig \ref{dll-family}).

\begin{figure}[!htbp]% [!hb] forces image to be placed at that position
	\centering
	\includegraphics[width=\columnwidth]{pics/dll-family.png}
	\caption{VirusTotal: Embedded \textit{DLL} family}
	\label{dll-family}
\end{figure}

\subsection{Sections}
The sections \textit{.text, .rdata, .data, .rsrc, .reloc and .bss} do not show any significant deviation from the ordinary in terms of virtual sizes, entropy as well as novelty.

\subsection{Analysis}
Much of the static analysis of the \textit{DLL} is thwarted by some anti-disassembly technique since \textit{Ghidra and Rizin/Radare2} could not corroborate with \textit{x64\_dbg} during debugging in terms of instruction alignment.
Due to this, not much static analysis could be performed during the current analysis.

Although, outputs from \textit{strings} command extracts some illuminating information nevertheless.
Screenshots in appendix of the strings illustrate that the \textit{DLL} is associated with the \textit{WarZone} too. This is evident from the presence of the string \textit{warzone160} as well as multiple imports that overlap with the previously analyzed binary.

\section{Dynamic Analysis}
		\subsection{Network Based Analysis}
			Attempted TCP connection to address \textit{195.140.214.82:6703} (Fig. \ref{wireshark}).

		\subsection{File System Based Analysis}
				\subsubsection{File System Changes}
				Opened file \textit{``:Zone.Identifier''}
				\subsubsection{Windows Registry Changes}
				\begin{itemize}
					\item Registry Key Set \textit{``Software\textbackslash Microsoft\textbackslash Windows\textbackslash CurrentVersion\textbackslash Internet\textbackslash Settings\textbackslash MaxConnectionsPer1\_0Server''} to \textit{4}
					\item Registry Key Set \textit{``Software\textbackslash Microsoft\textbackslash Windows\textbackslash CurrentVersion\textbackslash Internet\textbackslash Settings\textbackslash MaxConnectionsPerServer''} to \textit{4}
					\item Registry Add (expected) \textit{``SOFTWARE\textbackslash Microsoft\textbackslash Windows\textbackslash NT\textbackslash CurrentVersion\textbackslash Winlogon\textbackslash SpecialAccounts\textbackslash UserList''} (Fig \ref{reg2})
					\item Registry Add (expected) \textit{``Software\textbackslash Classes\textbackslash Folder\textbackslash shell\textbackslash open\textbackslash command''} (Fig. \ref{reg3})
				\end{itemize}

\newpage

\section{Indicators of Compromise}
			\subsection{Network Based}
			Attempted TCP connection to address \textit{195.140.214.82:6703}.
			\subsection{Host Based}
			\begin{itemize}
				\item Registry Key Set \textit{``Software\textbackslash Microsoft\textbackslash Windows\textbackslash CurrentVersion\textbackslash Internet\textbackslash Settings\textbackslash MaxConnectionsPer1\_0Server''} to \textit{4}
				\item Registry Key Set \textit{``Software\textbackslash Microsoft\textbackslash Windows\textbackslash CurrentVersion\textbackslash Internet\textbackslash Settings\textbackslash MaxConnectionsPerServer''} to \textit{4}
				\item Registry Add (expected) \textit{``SOFTWARE\textbackslash Microsoft\textbackslash Windows\textbackslash NT\textbackslash CurrentVersion\textbackslash Winlogon\textbackslash SpecialAccounts\textbackslash UserList''} (Fig \ref{reg2})
				\item Registry Add (expected) \textit{``Software\textbackslash Classes\textbackslash Folder\textbackslash shell\textbackslash open\textbackslash command''} (Fig. \ref{reg3})
				\item File Opened \textit{``:Zone.Identifier''} (Fig. \ref{zone})
			\end{itemize}
			\subsection{\textit{YARA} Rule}
			Visit \cite{ghub} for rule file if copying fails.
\lstinputlisting[language={},basicstyle=\tiny]{../rule.yar}
\newpage
\section{Appendix A: Screenshots}
\begin{figure}[!htbp]% [!hb] forces image to be placed at that position
	\centering
	\includegraphics[width=\columnwidth]{pics/callpop.png}
	\caption{Ghidra: Call/Pop technique to get \textit{EIP}}
	\label{callpop}
\end{figure}

\begin{figure}[!htbp]% [!hb] forces image to be placed at that position
	\centering
	\includegraphics[width=\columnwidth]{pics/shellcode-push.png}
	\caption{Ghidra: Shell Code call to \textit{build\_IAT\_and\_jump}}
	\label{shellcode-push}
\end{figure}

\begin{figure}[!htbp]% [!hb] forces image to be placed at that position
	\centering
	\includegraphics[width=\columnwidth]{pics/warzone.png}
	\caption{Ghidra: \textit{warzone160} string}
	\label{warzone}
\end{figure}

\begin{figure}[!htbp]% [!hb] forces image to be placed at that position
	\centering
	\includegraphics[width=\columnwidth]{pics/string1.png}
	\caption{Strings: Embedded \textit{DLL} strings 1}
	\label{string1}
\end{figure}

\begin{figure}[!htbp]% [!hb] forces image to be placed at that position
	\centering
	\includegraphics[width=\columnwidth]{pics/string2.png}
	\caption{Strings: Embedded \textit{DLL} strings 2}
	\label{string2}
\end{figure}

\begin{figure}[!htbp]% [!hb] forces image to be placed at that position
	\centering
	\includegraphics[width=\columnwidth]{pics/string3.png}
	\caption{Strings: Embedded \textit{DLL} strings 3}
	\label{string3}
\end{figure}

\begin{figure}[!htbp]% [!hb] forces image to be placed at that position
	\centering
	\includegraphics[width=\columnwidth]{pics/string4.png}
	\caption{Strings: Embedded \textit{DLL} strings 4}
	\label{string4}
\end{figure}

\begin{figure}[!htbp]% [!hb] forces image to be placed at that position
	\centering
	\includegraphics[width=\columnwidth]{pics/string5.png}
	\caption{Strings: Embedded \textit{DLL} strings 5}
	\label{string5}
\end{figure}

\begin{figure}[!htbp]% [!hb] forces image to be placed at that position
	\centering
	\includegraphics[width=\columnwidth]{pics/string6.png}
	\caption{Strings: Embedded \textit{DLL} strings 6}
	\label{string6}
\end{figure}

\begin{figure}[!htbp]% [!hb] forces image to be placed at that position
	\centering
	\includegraphics[width=\columnwidth]{pics/string7.png}
	\caption{Strings: Embedded \textit{DLL} strings 7}
	\label{string7}
\end{figure}

\begin{figure}[!htbp]% [!hb] forces image to be placed at that position
	\centering
	\includegraphics[width=\columnwidth]{pics/string8.png}
	\caption{Strings: Embedded \textit{DLL} strings 8}
	\label{string8}
\end{figure}

\begin{figure}[!htbp]% [!hb] forces image to be placed at that position
	\centering
	\includegraphics[width=\columnwidth]{pics/string9.png}
	\caption{Strings: Embedded \textit{DLL} strings 9}
	\label{string9}
\end{figure}

\begin{figure}[!htbp]% [!hb] forces image to be placed at that position
	\centering
	\includegraphics[width=\columnwidth]{pics/string10.png}
	\caption{Strings: Embedded \textit{DLL} strings 10}
	\label{string10}
\end{figure}

\begin{figure}[!htbp]% [!hb] forces image to be placed at that position
	\centering
	\includegraphics[width=\columnwidth]{pics/string11.png}
	\caption{Strings: Embedded \textit{DLL} strings 11}
	\label{string11}
\end{figure}

\begin{figure}[!htbp]% [!hb] forces image to be placed at that position
	\centering
	\includegraphics[width=\columnwidth]{pics/string12.png}
	\caption{Strings: Embedded \textit{DLL} strings 12}
	\label{string12}
\end{figure}

\begin{figure}[!htbp]% [!hb] forces image to be placed at that position
	\centering
	\includegraphics[width=\columnwidth]{pics/string13.png}
	\caption{Strings: Embedded \textit{DLL} strings 13}
	\label{string13}
\end{figure}

\begin{figure}[!htbp]% [!hb] forces image to be placed at that position
	\centering
	\includegraphics[width=\columnwidth]{pics/wireshark2.png}
	\caption{Wireshark: Connection to IP \textit{195.140.214.82} at port 6703}
	\label{wireshark}
\end{figure}

\begin{figure}[!htbp]% [!hb] forces image to be placed at that position
	\centering
	\includegraphics[width=\columnwidth]{pics/reg1.png}
	\caption{Ghidra: Registry Entry 1}
	\label{reg1}
\end{figure}

\begin{figure}[!htbp]% [!hb] forces image to be placed at that position
	\centering
	\includegraphics[width=\columnwidth]{pics/reg2.png}
	\caption{Ghidra: Registry Entry 2}
	\label{reg2}
\end{figure}

\begin{figure}[!htbp]% [!hb] forces image to be placed at that position
	\centering
	\includegraphics[width=\columnwidth]{pics/reg3.png}
	\caption{Ghidra: Registry Entry 3}
	\label{reg3}
\end{figure}

\begin{figure}[!htbp]% [!hb] forces image to be placed at that position
	\centering
	\includegraphics[width=\columnwidth]{pics/zone.png}
	\caption{Ghidra: \textit{``:Zone.Identifier''}}
	\label{zone}
\end{figure}

\newpage

\printbibliography
\end{document}